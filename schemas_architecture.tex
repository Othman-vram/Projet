\documentclass[10pt,a4paper]{article}
\usepackage[utf8]{inputenc}
\usepackage[french]{babel}
\usepackage{tikz}
\usepackage{geometry}
\geometry{margin=1.5cm}

% Définition des couleurs
\definecolor{inputcolor}{RGB}{52, 152, 219}
\definecolor{processcolor}{RGB}{46, 204, 113}
\definecolor{outputcolor}{RGB}{231, 76, 60}
\definecolor{toolcolor}{RGB}{155, 89, 182}
\definecolor{datacolor}{RGB}{241, 196, 15}

% Styles des nœuds
\tikzstyle{input} = [rectangle, rounded corners, minimum width=2.5cm, minimum height=0.8cm, text centered, draw=inputcolor, fill=inputcolor!20, thick]
\tikzstyle{process} = [rectangle, rounded corners, minimum width=2.5cm, minimum height=0.8cm, text centered, draw=processcolor, fill=processcolor!20, thick]
\tikzstyle{output} = [rectangle, rounded corners, minimum width=2.5cm, minimum height=0.8cm, text centered, draw=outputcolor, fill=outputcolor!20, thick]
\tikzstyle{tool} = [rectangle, rounded corners, minimum width=2cm, minimum height=0.6cm, text centered, draw=toolcolor, fill=toolcolor!20, thick]
\tikzstyle{data} = [ellipse, minimum width=2cm, minimum height=0.6cm, text centered, draw=datacolor, fill=datacolor!20, thick]
\tikzstyle{arrow} = [thick,->,>=stealth]
\tikzstyle{dasharrow} = [thick,->,>=stealth,dashed]

\begin{document}

\title{Schémas d'Architecture Technique\\Outil de Suture d'Images Histologiques}
\author{EL IDRISSI Othman}
\date{\today}

\maketitle

\section{Architecture de la Phase de Prétraitement}

\begin{figure}[htbp]
\centering
\begin{tikzpicture}[scale=0.9, every node/.style={scale=0.8}]

% Titre du schéma
\node[above] at (6,7) {\large \textbf{Pipeline de Prétraitement des Images Histologiques}};

% Ligne 1 - Entrées et outils
\node (svs) [input] at (0,5.5) {Image\\SVS/MRXS};
\node (qupath) [tool] at (3,5.5) {QuPath\\+ SAM};
\node (selection) [process] at (6,5.5) {Sélection\\Zones Tissu};
\node (geojson) [data] at (9,5.5) {Masque\\GeoJSON};

% Ligne 2 - Pipeline principal
\node (pipeline) [process] at (4.5,3.5) {Pipeline Python\\unified\_tissue\_pipeline.py};

% Ligne 3 - Sous-processus
\node (conversion) [process] at (1,1.5) {Conversion\\SVS → TIFF};
\node (maskgen) [process] at (4.5,1.5) {Génération\\Masque};
\node (extraction) [process] at (8,1.5) {Extraction\\RGBA};

% Ligne 4 - Sortie
\node (tiffout) [output] at (4.5,0) {TIFF Pyramidal\\RGBA Prétraité};

% Flèches horizontales ligne 1
\draw [arrow] (svs) -- (qupath);
\draw [arrow] (qupath) -- (selection);
\draw [arrow] (selection) -- (geojson);

% Flèches vers pipeline
\draw [arrow] (svs) -- (1.5,4.5) -- (3,3.5);
\draw [arrow] (geojson) -- (7.5,4.5) -- (6,3.5);

% Flèches du pipeline vers sous-processus
\draw [arrow] (pipeline) -- (conversion);
\draw [arrow] (pipeline) -- (maskgen);
\draw [arrow] (pipeline) -- (extraction);

% Flèches entre sous-processus
\draw [arrow] (conversion) -- (maskgen);
\draw [arrow] (maskgen) -- (extraction);

% Flèche finale
\draw [arrow] (extraction) -- (tiffout);

% Annotations
\node[text width=2cm, align=center] at (-1.5,3) {\tiny \textit{Formats:\\SVS, MRXS\\TIFF}};
\node[text width=2cm, align=center] at (10.5,3) {\tiny \textit{Segmentation\\automatique\\SAM}};
\node[text width=2.5cm, align=center] at (7.5,-0.5) {\tiny \textit{Fond transparent\\Multi-résolution\\Optimisé}};

\end{tikzpicture}
\caption{Architecture de la phase de prétraitement}
\end{figure}

\newpage

\section{Architecture de la Phase de Suture}

\begin{figure}[htbp]
\centering
\begin{tikzpicture}[scale=0.85, every node/.style={scale=0.75}]

% Titre
\node[above] at (6,8) {\large \textbf{Architecture Application de Suture (PyQt6)}};

% Entrée
\node (input) [input] at (6,7) {Fragments TIFF\\Pyramidaux};

% Couche chargement
\node (loader) [process] at (6,6) {Image Loader\\OpenSlide + tifffile};

% Architecture MVC - Modèle
\node (fragmgr) [process] at (2,4.5) {Fragment\\Manager};
\node (pointmgr) [process] at (2,3.5) {Point\\Manager};

% Architecture MVC - Vue
\node (mainwin) [process] at (6,4.5) {Main\\Window};
\node (canvas) [process] at (6,3.5) {Canvas\\Widget};
\node (controls) [process] at (6,2.5) {Control\\Panel};

% Architecture MVC - Contrôleur
\node (algo) [process] at (10,4.5) {Algorithmes\\Suture};
\node (export) [process] at (10,3.5) {Export\\Manager};

% Fonctionnalités
\node (manip) [tool] at (2,1) {Manipulation\\Fragments};
\node (align) [tool] at (6,1) {Alignement\\Manuel};
\node (stitch) [tool] at (10,1) {Suture\\Rigide};

% Sortie
\node (output) [output] at (6,0) {Image Finale\\TIFF Pyramidal};

% Flèches principales
\draw [arrow] (input) -- (loader);
\draw [arrow] (loader) -- (mainwin);

% Flèches MVC
\draw [arrow] (fragmgr) -- (mainwin);
\draw [arrow] (pointmgr) -- (mainwin);
\draw [arrow] (mainwin) -- (canvas);
\draw [arrow] (mainwin) -- (controls);
\draw [arrow] (algo) -- (mainwin);
\draw [arrow] (export) -- (mainwin);

% Flèches vers fonctionnalités
\draw [arrow] (controls) -- (manip);
\draw [arrow] (controls) -- (align);
\draw [arrow] (algo) -- (stitch);

% Flèche finale
\draw [arrow] (export) -- (output);

% Labels des groupes
\node[processcolor] at (2,5.2) {\small \textbf{Modèle}};
\node[toolcolor] at (6,5.2) {\small \textbf{Vue}};
\node[outputcolor] at (10,5.2) {\small \textbf{Contrôleur}};

% Annotations
\node[text width=2cm, align=center] at (0,3) {\tiny \textit{Gestion\\état\\fragments}};
\node[text width=2cm, align=center] at (12,3) {\tiny \textit{SIFT\\L-BFGS-B\\Optimisation}};

\end{tikzpicture}
\caption{Architecture de l'application de suture}
\end{figure}

\newpage

\section{Flux de Données Global}

\begin{figure}[htbp]
\centering
\begin{tikzpicture}[scale=1, every node/.style={scale=0.8}]

% Titre
\node[above] at (6,6.5) {\large \textbf{Flux de Données Global du Système}};

% Phase 1
\node (micro) [input] at (1,5) {Microscope\\Scanner};
\node (raw) [data] at (4,5) {Image Brute\\SVS/MRXS};

% Phase 2
\node (preproc) [process] at (7,5) {Prétraitement\\Pipeline};
\node (frags) [data] at (10,5) {Fragments\\TIFF RGBA};

% Phase 3
\node (app) [process] at (7,3) {Application\\Suture};
\node (final) [output] at (10,3) {Image Finale\\Reconstituée};

% Phase 4
\node (analysis) [tool] at (7,1) {Analyse\\TEP Margins};
\node (diag) [output] at (10,1) {Diagnostic\\Médical};

% Flèches principales
\draw [arrow] (micro) -- (raw);
\draw [arrow] (raw) -- (preproc);
\draw [arrow] (preproc) -- (frags);
\draw [arrow] (frags) -- (app);
\draw [arrow] (app) -- (final);
\draw [arrow] (final) -- (analysis);
\draw [arrow] (analysis) -- (diag);

% Labels des phases
\node[text width=1.5cm, align=center] at (1,3.5) {\small \textbf{Phase 1}\\Acquisition};
\node[text width=1.5cm, align=center] at (4,3.5) {\small \textbf{Phase 2}\\Prétraitement};
\node[text width=1.5cm, align=center] at (7,2) {\small \textbf{Phase 3}\\Suture};
\node[text width=1.5cm, align=center] at (10,2) {\small \textbf{Phase 4}\\Application};

% Annotations techniques
\node[text width=2cm, align=center] at (1,6) {\tiny \textit{0.25 µm/pixel\\Format propriétaire}};
\node[text width=2cm, align=center] at (11.5,5) {\tiny \textit{Fond transparent\\Multi-résolution}};
\node[text width=2cm, align=center] at (11.5,3) {\tiny \textit{Haute définition\\Métadonnées}};
\node[text width=2cm, align=center] at (11.5,1) {\tiny \textit{Marges chirurgicales\\Micro-TEP}};

\end{tikzpicture}
\caption{Flux de données global du système}
\end{figure}

\newpage

\section{Diagramme de Classes Simplifié}

\begin{figure}[htbp]
\centering
\begin{tikzpicture}[scale=0.8, every node/.style={scale=0.7}]

% Titre
\node[above] at (6,7) {\large \textbf{Diagramme de Classes Principal}};

% Classes principales avec positionnement absolu
\node (fragment) [process, text width=2.5cm] at (2,5) {
    \textbf{Fragment}\\
    \rule{2.5cm}{0.4pt}\\
    - id: str\\
    - image\_data\\
    - x, y: float\\
    - rotation: float\\
    \rule{2.5cm}{0.4pt}\\
    + get\_transformed()\\
    + contains\_point()
};

\node (fragmgr) [process, text width=2.5cm] at (6,5) {
    \textbf{FragmentManager}\\
    \rule{2.5cm}{0.4pt}\\
    - fragments: Dict\\
    - selected\_id: str\\
    \rule{2.5cm}{0.4pt}\\
    + add\_fragment()\\
    + rotate\_fragment()\\
    + translate\_fragment()
};

\node (canvas) [process, text width=2.5cm] at (10,5) {
    \textbf{CanvasWidget}\\
    \rule{2.5cm}{0.4pt}\\
    - zoom: float\\
    - pan\_x, pan\_y\\
    - fragments: List\\
    \rule{2.5cm}{0.4pt}\\
    + paintEvent()\\
    + mousePressEvent()
};

\node (loader) [process, text width=2.5cm] at (2,2) {
    \textbf{ImageLoader}\\
    \rule{2.5cm}{0.4pt}\\
    - supported\_formats\\
    \rule{2.5cm}{0.4pt}\\
    + load\_image()\\
    + get\_pyramid\_info()
};

\node (exporter) [process, text width=2.5cm] at (10,2) {
    \textbf{PyramidalExporter}\\
    \rule{2.5cm}{0.4pt}\\
    \rule{2.5cm}{0.4pt}\\
    + export\_pyramidal()\\
    + create\_composite()
};

% Relations avec positionnement précis
\draw [arrow] (fragmgr) -- (fragment);
\draw [arrow] (canvas) -- (fragmgr);
\draw [arrow] (loader) -- (fragment);
\draw [arrow] (exporter) -- (fragmgr);

% Labels des relations
\node at (4,5.2) {\tiny manages};
\node at (8,5.2) {\tiny displays};
\node at (2,3.5) {\tiny creates};
\node at (8,3.5) {\tiny exports};

\end{tikzpicture}
\caption{Diagramme de classes simplifié}
\end{figure}

\end{document}