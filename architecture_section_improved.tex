\subsection{Architecture générale du système}

La solution développée s'articule autour de deux composants principaux : un pipeline de prétraitement et une application desktop de suture manuelle. Cette architecture modulaire permet une séparation claire des responsabilités et facilite la maintenance et l'évolution du système.

% Schéma d'architecture générale - placé en premier pour donner la vue d'ensemble
\begin{figure}[H]
\centering
\begin{tikzpicture}[node distance=2cm, every node/.style={scale=0.8}]

% Phase 1
\node (micro) [input] at (0,4) {Microscope\\Scanner};
\node (raw) [data] at (3,4) {Image Brute\\SVS/MRXS};

% Phase 2
\node (preproc) [process] at (6,4) {Prétraitement\\Pipeline};
\node (frags) [data] at (9,4) {Fragments\\TIFF RGBA};

% Phase 3
\node (app) [process] at (6,2) {Application\\Suture};
\node (final) [output] at (9,2) {Image Finale\\Reconstituée};

% Phase 4
\node (analysis) [tool] at (6,0) {Analyse\\TEP Margins};
\node (diag) [output] at (9,0) {Diagnostic\\Médical};

% Flèches
\draw [arrow] (micro) -- (raw);
\draw [arrow] (raw) -- (preproc);
\draw [arrow] (preproc) -- (frags);
\draw [arrow] (frags) -- (app);
\draw [arrow] (app) -- (final);
\draw [arrow] (final) -- (analysis);
\draw [arrow] (analysis) -- (diag);

% Labels des phases
\node[text width=1.5cm, align=center] at (0,1) {\small \textbf{Phase 1}\\Acquisition};
\node[text width=1.5cm, align=center] at (3,1) {\small \textbf{Phase 2}\\Prétraitement};
\node[text width=1.5cm, align=center] at (6,1) {\small \textbf{Phase 3}\\Suture};
\node[text width=1.5cm, align=center] at (9,1) {\small \textbf{Phase 4}\\Application};

\end{tikzpicture}
\caption{Flux de données global du système développé}
\end{figure}

\subsubsection{Module de prétraitement}

Le pipeline transforme les images SVS/MRXS en fragments TIFF pyramidaux exploitables. Il utilise QuPath avec le plugin SAM pour la segmentation assistée, puis génère automatiquement les fichiers RGBA avec fond transparent.

% Schéma d'architecture du prétraitement - placé logiquement après l'explication
\begin{figure}[H]
\centering
\begin{tikzpicture}[node distance=1.5cm, every node/.style={scale=0.7}]

% Ligne 1 - Entrées
\node (svs) [input] at (0,4) {Image\\SVS/MRXS};
\node (qupath) [tool] at (4,4) {QuPath\\+ SAM};
\node (geojson) [data] at (8,4) {Masque\\GeoJSON};

% Ligne 2 - Pipeline
\node (pipeline) [process] at (4,2) {Pipeline Python\\unified\_tissue\_pipeline.py};

% Ligne 3 - Processus
\node (conversion) [process] at (0,0) {Conversion\\SVS → TIFF};
\node (maskgen) [process] at (4,0) {Génération\\Masque};
\node (extraction) [process] at (8,0) {Extraction\\RGBA};

% Ligne 4 - Sortie
\node (tiffout) [output] at (4,-2) {TIFF Pyramidal\\RGBA Prétraité};

% Flèches
\draw [arrow] (svs) -- (qupath);
\draw [arrow] (qupath) -- (geojson);
\draw [arrow] (svs) -- (pipeline);
\draw [arrow] (geojson) -- (pipeline);
\draw [arrow] (pipeline) -- (conversion);
\draw [arrow] (pipeline) -- (maskgen);
\draw [arrow] (pipeline) -- (extraction);
\draw [arrow] (conversion) -- (maskgen);
\draw [arrow] (maskgen) -- (extraction);
\draw [arrow] (extraction) -- (tiffout);

\end{tikzpicture}
\caption{Architecture de la phase de prétraitement}
\end{figure}

% Interface QuPath - placée ici car elle illustre directement la segmentation
\begin{figure}[H]
\centering
\includegraphics[width=0.8\textwidth]{images/qupath_sam_segmentation_screenshot.png}
\caption{Interface QuPath avec plugin SAM pour la segmentation des tissus}
\end{figure}

% Pipeline en exécution - placée ici car elle montre le processus automatisé
\begin{figure}[H]
\centering
\includegraphics[width=0.8\textwidth]{images/pipeline_execution_screenshot.png}
\caption{Exécution du pipeline de prétraitement avec barres de progression}
\end{figure}

\subsubsection{Module de suture interactive}

L'application desktop suit une architecture MVC avec PyQt6, permettant la manipulation manuelle des fragments prétraités pour reconstituer l'image histologique complète.

% Architecture MVC - placée avant l'interface pour expliquer la structure
\begin{figure}[H]
\centering
\begin{tikzpicture}[node distance=1.5cm, every node/.style={scale=0.7}]

% Entrée
\node (input) [input] at (4,6) {Fragments TIFF\\Pyramidaux};

% Chargement
\node (loader) [process] at (4,4.5) {Image Loader\\OpenSlide + tifffile};

% Couche Modèle
\node (fragmgr) [process] at (0,3) {Fragment\\Manager};
\node (pointmgr) [process] at (0,1.5) {Point\\Manager};

% Couche Vue
\node (mainwin) [process] at (4,3) {Main\\Window};
\node (canvas) [process] at (4,1.5) {Canvas\\Widget};

% Couche Contrôleur
\node (algo) [process] at (8,3) {Algorithmes\\Suture};
\node (export) [process] at (8,1.5) {Export\\Manager};

% Sortie
\node (output) [output] at (4,0) {Image Finale\\TIFF Pyramidal};

% Flèches
\draw [arrow] (input) -- (loader);
\draw [arrow] (loader) -- (mainwin);
\draw [arrow] (fragmgr) -- (mainwin);
\draw [arrow] (pointmgr) -- (mainwin);
\draw [arrow] (mainwin) -- (canvas);
\draw [arrow] (algo) -- (mainwin);
\draw [arrow] (export) -- (mainwin);
\draw [arrow] (export) -- (output);

% Labels
\node[processcolor] at (0,3.8) {\small \textbf{Modèle}};
\node[toolcolor] at (4,3.8) {\small \textbf{Vue}};
\node[outputcolor] at (8,3.8) {\small \textbf{Contrôleur}};

\end{tikzpicture}
\caption{Architecture MVC de l'application de suture}
\end{figure}

% Interface principale - placée après l'architecture pour montrer le résultat concret
\begin{figure}[H]
\centering
\includegraphics[width=0.9\textwidth]{images/interface_principale_screenshot.png}
\caption{Interface principale avec zones fonctionnelles}
\end{figure}

% Canvas avec fragments - illustre la visualisation des données
\begin{figure}[H]
\centering
\includegraphics[width=0.8\textwidth]{images/canvas_fragments_screenshot.png}
\caption{Visualisation des fragments sur le canvas principal}
\end{figure}

\subsubsection{Fonctionnalités de manipulation}

L'application offre des outils complets de manipulation individuelle et de groupe :

% Sélection de groupe - placée ici car elle illustre directement la manipulation
\begin{figure}[H]
\centering
\begin{subfigure}{0.48\textwidth}
\includegraphics[width=\textwidth]{images/selection_rectangle_screenshot.png}
\caption{Sélection rectangle multiple}
\end{subfigure}
\hfill
\begin{subfigure}{0.48\textwidth}
\includegraphics[width=\textwidth]{images/panneau_groupe_screenshot.png}
\caption{Contrôles de groupe}
\end{subfigure}
\caption{Outils de sélection et manipulation de groupe}
\end{figure}

Le système de points étiquetés permet un alignement précis basé sur des correspondances anatomiques définies manuellement :

% Points étiquetés - placée ici car elle illustre cette fonctionnalité spécifique
\begin{figure}[H]
\centering
\includegraphics[width=0.8\textwidth]{images/points_etiquetes_screenshot.png}
\caption{Points étiquetés pour correspondances entre fragments}
\end{figure}

\subsubsection{Exportation pyramidale}

L'exportation constitue l'étape finale permettant de générer les images reconstituées pour intégration dans le workflow TEP Margins :

% Dialogues d'export - placés ici car ils illustrent directement l'exportation
\begin{figure}[H]
\centering
\begin{subfigure}{0.48\textwidth}
\includegraphics[width=\textwidth]{images/dialogue_export_screenshot.png}
\caption{Options d'exportation}
\end{subfigure}
\hfill
\begin{subfigure}{0.48\textwidth}
\includegraphics[width=\textwidth]{images/selection_niveaux_screenshot.png}
\caption{Sélection niveaux pyramidaux}
\end{subfigure}
\caption{Interface d'exportation avec contrôle des niveaux de résolution}
\end{figure}

L'application propose deux formats principaux : PNG pour les aperçus rapides et TIFF pyramidal pour l'analyse approfondie avec préservation de la structure multi-résolution.