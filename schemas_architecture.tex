\documentclass[12pt,a4paper]{article}
\usepackage[utf8]{inputenc}
\usepackage[french]{babel}
\usepackage{tikz}
\usepackage{geometry}
\usepackage{xcolor}

\geometry{margin=2cm}

\usetikzlibrary{shapes.geometric, arrows, positioning, fit, backgrounds}
\usetikzlibrary{decorations.pathreplacing}
\usetikzlibrary{calc}
\usetikzlibrary{shadows}

% Définition des couleurs
\definecolor{inputcolor}{RGB}{52, 152, 219}
\definecolor{processcolor}{RGB}{46, 204, 113}
\definecolor{outputcolor}{RGB}{231, 76, 60}
\definecolor{toolcolor}{RGB}{155, 89, 182}
\definecolor{datacolor}{RGB}{241, 196, 15}

% Styles pour les boîtes
\tikzstyle{input} = [rectangle, rounded corners, minimum width=3cm, minimum height=1cm, text centered, draw=inputcolor, fill=inputcolor!20, thick]
\tikzstyle{process} = [rectangle, rounded corners, minimum width=3cm, minimum height=1cm, text centered, draw=processcolor, fill=processcolor!20, thick]
\tikzstyle{output} = [rectangle, rounded corners, minimum width=3cm, minimum height=1cm, text centered, draw=outputcolor, fill=outputcolor!20, thick]
\tikzstyle{tool} = [rectangle, rounded corners, minimum width=2.5cm, minimum height=0.8cm, text centered, draw=toolcolor, fill=toolcolor!20, thick]
\tikzstyle{data} = [ellipse, minimum width=2.5cm, minimum height=1cm, text centered, draw=datacolor, fill=datacolor!20, thick]
\tikzstyle{arrow} = [thick,->,>=stealth]
\tikzstyle{dasharrow} = [thick,->,>=stealth,dashed]

\begin{document}

\title{Schémas d'Architecture - Pipeline de Traitement d'Images Histologiques}
\author{EL IDRISSI Othman}
\date{\today}
\maketitle

\section{Architecture de la Phase de Prétraitement}

\begin{figure}[h!]
\centering
\begin{tikzpicture}[node distance=2cm]

% Titre du schéma
\node[above] at (0,8) {\Large \textbf{Pipeline de Prétraitement des Images Histologiques}};

% Entrées
\node (svs) [input] at (-6,6) {Image SVS/MRXS\\(Lame scannée)};
\node (qupath) [tool] at (-2,6) {QuPath\\+ Plugin SAM};

% Processus de segmentation
\node (selection) [process] at (2,6) {Sélection\\Zones de Tissu};
\node (geojson) [data] at (6,6) {Masque\\GeoJSON};

% Pipeline Python
\node (pipeline) [process] at (0,3.5) {Pipeline Python\\unified\_tissue\_pipeline.py};

% Sous-processus du pipeline
\node (conversion) [process] at (-4,1.5) {Conversion\\SVS → TIFF};
\node (maskgen) [process] at (0,1.5) {Génération\\Masque Pyramidal};
\node (extraction) [process] at (4,1.5) {Extraction\\Tissus RGBA};

% Sortie finale
\node (tiffout) [output] at (0,-1) {TIFF Pyramidal\\RGBA Prétraité};

% Flèches principales
\draw [arrow] (svs) -- (qupath);
\draw [arrow] (qupath) -- (selection);
\draw [arrow] (selection) -- (geojson);

% Flèches vers le pipeline
\draw [arrow] (svs) -- (-3,4.5) -- (-1.5,3.5);
\draw [arrow] (geojson) -- (3,4.5) -- (1.5,3.5);

% Flèches du pipeline
\draw [arrow] (pipeline) -- (conversion);
\draw [arrow] (pipeline) -- (maskgen);
\draw [arrow] (pipeline) -- (extraction);

% Flèches entre sous-processus
\draw [arrow] (conversion) -- (maskgen);
\draw [arrow] (maskgen) -- (extraction);

% Flèche finale
\draw [arrow] (extraction) -- (tiffout);

% Annotations
\node[text width=3cm, align=center] at (-8,3) {\small \textit{Formats supportés:\\SVS, MRXS, TIFF}};
\node[text width=3cm, align=center] at (8,3) {\small \textit{Segmentation\\automatique\\avec SAM}};
\node[text width=4cm, align=center] at (6,-1) {\small \textit{Fond transparent\\Tissus préservés\\Multi-résolution}};

% Cadre pour le pipeline
\draw[dashed, thick, processcolor] (-5.5,0.5) rectangle (5.5,2.5);
\node[processcolor] at (0,2.8) {\textbf{Étapes du Pipeline Unifié}};

\end{tikzpicture}
\caption{Architecture de la phase de prétraitement}
\end{figure}

\newpage

\section{Architecture de la Phase de Suture}

\begin{figure}[h!]
\centering
\begin{tikzpicture}[node distance=2cm]

% Titre du schéma
\node[above] at (0,9) {\Large \textbf{Architecture de l'Application de Suture Manuelle}};

% Entrée
\node (tiffinput) [input] at (0,7.5) {Fragments TIFF\\Pyramidaux Prétraités};

% Couche de chargement
\node (loader) [process] at (0,6) {Image Loader\\(OpenSlide + tifffile)};

% Architecture MVC
\node (mvc) [above] at (0,4.8) {\textbf{Architecture MVC (PyQt6)}};

% Modèle
\node (fragmanager) [process] at (-4,4) {Fragment\\Manager};
\node (pointmanager) [process] at (-4,2.5) {Point\\Manager};

% Vue
\node (mainwindow) [process] at (0,4) {Main Window};
\node (canvas) [process] at (0,2.5) {Canvas Widget\\(Visualisation)};
\node (controls) [process] at (0,1) {Control Panel\\(Manipulation)};

% Contrôleur
\node (algorithms) [process] at (4,4) {Algorithmes\\de Suture};
\node (exporters) [process] at (4,2.5) {Export\\Manager};

% Fonctionnalités principales
\node (features) [above] at (0,-0.5) {\textbf{Fonctionnalités Principales}};

\node (manipulation) [tool] at (-3,-1.5) {Manipulation\\Fragments};
\node (alignment) [tool] at (0,-1.5) {Alignement\\Manuel};
\node (stitching) [tool] at (3,-1.5) {Suture\\Rigide};

% Sortie
\node (finaloutput) [output] at (0,-3.5) {Image Finale\\TIFF Pyramidal\\Haute Définition};

% Flèches principales
\draw [arrow] (tiffinput) -- (loader);
\draw [arrow] (loader) -- (fragmanager);
\draw [arrow] (loader) -- (mainwindow);

% Flèches MVC
\draw [arrow] (fragmanager) -- (mainwindow);
\draw [arrow] (pointmanager) -- (mainwindow);
\draw [arrow] (mainwindow) -- (canvas);
\draw [arrow] (mainwindow) -- (controls);
\draw [arrow] (canvas) -- (algorithms);
\draw [arrow] (controls) -- (algorithms);
\draw [arrow] (algorithms) -- (exporters);

% Flèches bidirectionnelles pour l'interaction
\draw [dasharrow] (canvas) -- (fragmanager);
\draw [dasharrow] (controls) -- (fragmanager);
\draw [dasharrow] (canvas) -- (pointmanager);

% Flèches vers fonctionnalités
\draw [arrow] (canvas) -- (manipulation);
\draw [arrow] (canvas) -- (alignment);
\draw [arrow] (algorithms) -- (stitching);

% Flèche finale
\draw [arrow] (exporters) -- (finaloutput);

% Annotations techniques
\node[text width=2.5cm, align=center] at (-6.5,3) {\small \textit{Gestion état\\fragments\\transformations}};
\node[text width=2.5cm, align=center] at (6.5,3) {\small \textit{SIFT features\\Optimisation\\L-BFGS-B}};
\node[text width=3cm, align=center] at (6,-3.5) {\small \textit{Multi-niveaux\\Compression LZW\\Métadonnées}};

% Cadres pour grouper les composants
\draw[dashed, thick, processcolor] (-5,1.5) rectangle (-3,4.5);
\node[processcolor] at (-4,4.8) {\textbf{Modèle}};

\draw[dashed, thick, toolcolor] (-1,0.5) rectangle (1,4.5);
\node[toolcolor] at (0,4.8) {\textbf{Vue}};

\draw[dashed, thick, outputcolor] (3,1.5) rectangle (5,4.5);
\node[outputcolor] at (4,4.8) {\textbf{Contrôleur}};

\end{tikzpicture}
\caption{Architecture de l'application de suture manuelle}
\end{figure}

\newpage

\section{Flux de Données Global}

\begin{figure}[h!]
\centering
\begin{tikzpicture}[node distance=1.5cm]

% Titre
\node[above] at (0,8) {\Large \textbf{Flux de Données Global du Système}};

% Phase 1 - Acquisition
\node (microscope) [input] at (-6,6) {Microscope\\Scanner};
\node (rawimage) [data] at (-3,6) {Image Brute\\SVS/MRXS};

% Phase 2 - Prétraitement
\node (preprocessing) [process] at (0,6) {Prétraitement\\Pipeline};
\node (fragments) [data] at (3,6) {Fragments\\TIFF RGBA};

% Phase 3 - Suture
\node (stitching) [process] at (0,3) {Application\\Suture};
\node (finalimage) [output] at (3,3) {Image Finale\\Reconstituée};

% Phase 4 - Utilisation
\node (analysis) [tool] at (0,0) {Analyse\\TEP Margins};
\node (diagnosis) [output] at (3,0) {Diagnostic\\Médical};

% Flèches principales
\draw [arrow] (microscope) -- (rawimage);
\draw [arrow] (rawimage) -- (preprocessing);
\draw [arrow] (preprocessing) -- (fragments);
\draw [arrow] (fragments) -- (stitching);
\draw [arrow] (stitching) -- (finalimage);
\draw [arrow] (finalimage) -- (analysis);
\draw [arrow] (analysis) -- (diagnosis);

% Annotations de phases
\node[text width=2cm, align=center] at (-6,4.5) {\small \textbf{Phase 1}\\Acquisition};
\node[text width=2cm, align=center] at (0,4.5) {\small \textbf{Phase 2}\\Prétraitement};
\node[text width=2cm, align=center] at (0,1.5) {\small \textbf{Phase 3}\\Suture};
\node[text width=2cm, align=center] at (0,-1.5) {\small \textbf{Phase 4}\\Application};

% Détails techniques
\node[text width=3cm, align=center] at (-6,7.5) {\tiny \textit{Résolution:\\0.25 µm/pixel\\Format propriétaire}};
\node[text width=3cm, align=center] at (6,6) {\tiny \textit{Fond transparent\\Multi-résolution\\Optimisé}};
\node[text width=3cm, align=center] at (6,3) {\tiny \textit{Haute définition\\Métadonnées\\Exportable}};
\node[text width=3cm, align=center] at (6,0) {\tiny \textit{Marges chirurgicales\\Micro-TEP\\Corrélation}};

\end{tikzpicture}
\caption{Flux de données global du système}
\end{figure}

\section{Diagramme de Classes Simplifié}

\begin{figure}[h!]
\centering
\begin{tikzpicture}[node distance=2cm]

% Titre
\node[above] at (0,7) {\Large \textbf{Diagramme de Classes Principal}};

% Classes principales
\node (fragment) [process, text width=3cm] at (-4,5) {
    \textbf{Fragment}\\
    \rule{3cm}{0.4pt}\\
    - id: str\\
    - image\_data: ndarray\\
    - x, y: float\\
    - rotation: float\\
    - visible: bool\\
    \rule{3cm}{0.4pt}\\
    + get\_transformed\_image()\\
    + contains\_point()
};

\node (fragmanager) [process, text width=3cm] at (0,5) {
    \textbf{FragmentManager}\\
    \rule{3cm}{0.4pt}\\
    - fragments: Dict\\
    - selected\_id: str\\
    \rule{3cm}{0.4pt}\\
    + add\_fragment()\\
    + rotate\_fragment()\\
    + translate\_fragment()
};

\node (canvas) [process, text width=3cm] at (4,5) {
    \textbf{CanvasWidget}\\
    \rule{3cm}{0.4pt}\\
    - zoom: float\\
    - pan\_x, pan\_y: float\\
    - fragments: List\\
    \rule{3cm}{0.4pt}\\
    + paintEvent()\\
    + mousePressEvent()
};

\node (imageloader) [process, text width=3cm] at (-4,2) {
    \textbf{ImageLoader}\\
    \rule{3cm}{0.4pt}\\
    - supported\_formats\\
    \rule{3cm}{0.4pt}\\
    + load\_image()\\
    + get\_pyramid\_info()
};

\node (exporter) [process, text width=3cm] at (4,2) {
    \textbf{PyramidalExporter}\\
    \rule{3cm}{0.4pt}\\
    \rule{3cm}{0.4pt}\\
    + export\_pyramidal\_tiff()\\
    + create\_level\_composite()
};

% Relations
\draw [arrow] (fragmanager) -- (fragment);
\draw [arrow] (canvas) -- (fragmanager);
\draw [arrow] (imageloader) -- (fragment);
\draw [arrow] (exporter) -- (fragmanager);

% Labels des relations
\node at (-2,5.3) {\tiny manages};
\node at (2,5.3) {\tiny displays};
\node at (-4,3.5) {\tiny creates};
\node at (4,3.5) {\tiny exports};

\end{tikzpicture}
\caption{Diagramme de classes simplifié}
\end{figure}

\end{document}