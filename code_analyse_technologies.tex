% Code à insérer dans la section 3.2 "Analyse des solutions existantes"
% Remplacez tout le contenu de cette section par le code ci-dessous

\subsection{Méthodologie d'Évaluation}

\begin{tcolorbox}[colback=TechBlue!10, colframe=TechBlue, title=Critères d'Évaluation]
\begin{itemize}[leftmargin=*]
    \item \textbf{Scalabilité} : Capacité à traiter un nombre variable de fragments
    \item \textbf{Formats supportés} : Compatibilité avec SVS, MRXS, TIFF pyramidal
    \item \textbf{Algorithmes} : Robustesse des méthodes de suture
    \item \textbf{Interface utilisateur} : Ergonomie et facilité d'utilisation
    \item \textbf{Maintenance} : État de développement et support communautaire
    \item \textbf{Déploiement} : Facilité d'installation en environnement clinique
\end{itemize}
\end{tcolorbox}

\subsubsection{PyThostitcher}

\begin{techbox}{TechBlue}{PyThostitcher - Outil Python de Suture SIFT}

\textbf{Développeur :} Communauté Python scientifique \\
\textbf{Langage :} Python \\
\textbf{Licence :} Open Source

\vspace{0.5cm}

\begin{tabularx}{\textwidth}{|X|X|}
\hline
\rowcolor{LightGray}
\textbf{Fonctionnement} & \textbf{Architecture Technique} \\
\hline
Utilise des descripteurs SIFT (Scale-Invariant Feature Transform) pour détecter des points d'intérêt dans les images & 
\begin{itemize}[nosep]
\item Détection de caractéristiques SIFT
\item Algorithmes de correspondance
\item Optimisation globale des positions
\item Interface Python native
\end{itemize} \\
\hline
\end{tabularx}

\vspace{0.5cm}

\textbf{Avantages identifiés :}
\begin{itemize}[leftmargin=*]
    \pro{Algorithmes SIFT robustes et éprouvés}
    \pro{Implémentation Python moderne}
    \pro{Optimisation globale des correspondances}
    \pro{Documentation technique disponible}
\end{itemize}

\textbf{Limitations critiques :}
\begin{itemize}[leftmargin=*]
    \con{Traitement limité à 2-4 fragments maximum}
    \con{Contrainte architecturale non extensible}
    \con{Inadapté aux cas complexes (>5 fragments)}
    \con{Pas de support des formats médicaux spécialisés}
\end{itemize}

\begin{center}
\textbf{\textcolor{DangerRed}{DÉCISION : REJETÉ}}\\
\textit{Limitations de scalabilité incompatibles avec les besoins}
\end{center}

\end{techbox}

\subsubsection{HistoStitcher}

\begin{techbox}{WarningOrange}{HistoStitcher - Solution MATLAB Historique}

\textbf{Développeur :} Laboratoire de recherche académique \\
\textbf{Plateforme :} MATLAB \\
\textbf{Statut :} Obsolète (non maintenu)

\vspace{0.5cm}

\begin{tabularx}{\textwidth}{|X|X|}
\hline
\rowcolor{LightGray}
\textbf{Approche Technique} & \textbf{Spécialisation Histologique} \\
\hline
Implémente des algorithmes de corrélation croisée et de transformation affine spécialement conçus pour l'imagerie histologique &
\begin{itemize}[nosep]
\item Interface graphique MATLAB
\item Algorithmes de corrélation croisée
\item Transformations affines optimisées
\item Calibration pour tissus biologiques
\end{itemize} \\
\hline
\end{tabularx}

\vspace{0.5cm}

\textbf{Points positifs historiques :}
\begin{itemize}[leftmargin=*]
    \pro{Spécialisé pour l'imagerie histologique}
    \pro{Algorithmes de corrélation éprouvés}
    \pro{Interface graphique intégrée}
\end{itemize}

\textbf{Obstacles majeurs :}
\begin{itemize}[leftmargin=*]
    \con{Outil ancien sans maintenance active}
    \con{Dépendance MATLAB coûteuse et restrictive}
    \con{Incompatibilité avec formats SVS/MRXS modernes}
    \con{Déploiement clinique complexe}
    \con{Pas d'évolution technologique récente}
\end{itemize}

\begin{center}
\textbf{\textcolor{DangerRed}{DÉCISION : REJETÉ}}\\
\textit{Obsolescence technique et limitations de déploiement}
\end{center}

\end{techbox}

\subsubsection{ASHLAR}

\begin{techbox}{SuccessGreen}{ASHLAR - Solution Harvard Medical School}

\textbf{Développeur :} Laboratory of Systems Pharmacology, Harvard Medical School \\
\textbf{Nom complet :} Alignment by Simultaneous Harmonization of Layer/Adjacency Registration \\
\textbf{Statut :} Activement développé et maintenu

\vspace{0.5cm}

\begin{tabularx}{\textwidth}{|X|X|}
\hline
\rowcolor{LightGray}
\textbf{Technologies Avancées} & \textbf{Capacités Biomédicales} \\
\hline
Algorithmes sophistiqués de détection de caractéristiques avec optimisation globale multi-échelle &
\begin{itemize}[nosep]
\item Support natif formats biomédicaux
\item Correction d'illumination automatique
\item Algorithmes de déformation robustes
\item Pipeline de traitement optimisé
\end{itemize} \\
\hline
\end{tabularx}

\vspace{0.5cm}

\textbf{Excellence technique :}
\begin{itemize}[leftmargin=*]
    \pro{Algorithmes de pointe développés par Harvard}
    \pro{Support natif des formats d'imagerie biomédicale}
    \pro{Correction automatique d'illumination et déformation}
    \pro{Optimisation globale multi-échelle}
    \pro{Documentation scientifique complète}
    \pro{Maintenance active et évolutions régulières}
\end{itemize}

\textbf{Contrainte fondamentale :}
\begin{itemize}[leftmargin=*]
    \con{Nécessite des zones de chevauchement entre fragments}
    \con{Inadapté aux tissus découpés physiquement}
    \con{Algorithmes basés sur la détection de correspondances}
    \con{Incompatible avec notre contexte sans recouvrement}
\end{itemize}

\begin{center}
\textbf{\textcolor{WarningOrange}{DÉCISION : REJETÉ}}\\
\textit{Excellente technologie mais inadaptée à nos données spécifiques}
\end{center}

\end{techbox}

\subsubsection{FIJI/ImageJ}

\begin{techbox}{InfoGray}{FIJI - Distribution ImageJ Enrichie}

\textbf{Nom complet :} Fiji Is Just ImageJ \\
\textbf{Écosystème :} Plateforme open-source avec plugins communautaires \\
\textbf{Spécialisation :} Analyse d'images scientifiques

\vspace{0.5cm}

\begin{tabularx}{\textwidth}{|X|X|}
\hline
\rowcolor{LightGray}
\textbf{Architecture Modulaire} & \textbf{Plugins de Suture} \\
\hline
Plateforme extensible avec écosystème de plugins spécialisés pour l'analyse d'images scientifiques &
\begin{itemize}[nosep]
\item Plugin "Stitching" intégré
\item Algorithmes phase-correlation
\item Interface graphique complète
\item Extensibilité par plugins
\end{itemize} \\
\hline
\end{tabularx}

\vspace{0.5cm}

\textbf{Avantages de l'écosystème :}
\begin{itemize}[leftmargin=*]
    \pro{Plateforme mature et stable}
    \pro{Large communauté scientifique}
    \pro{Plugins spécialisés nombreux}
    \pro{Interface utilisateur familière}
    \pro{Documentation extensive}
\end{itemize}

\textbf{Limitations pour notre usage :}
\begin{itemize}[leftmargin=*]
    \con{Même problématique qu'ASHLAR : nécessite des zones communes}
    \con{Algorithmes phase-correlation inadaptés sans chevauchement}
    \con{Pas d'adaptation possible aux tissus découpés}
    \con{Interface généraliste non optimisée pour notre cas d'usage}
\end{itemize}

\begin{center}
\textbf{\textcolor{DangerRed}{DÉCISION : REJETÉ}}\\
\textit{Mêmes limitations fondamentales qu'ASHLAR}
\end{center}

\end{techbox}

\subsubsection{Plugin Napari}

\begin{techbox}{TechBlue}{Napari - Plateforme de Visualisation Moderne}

\textbf{Technologie :} Python, Qt, OpenGL \\
\textbf{Spécialisation :} Visualisation d'images multi-dimensionnelles \\
\textbf{Architecture :} Modulaire avec système de plugins

\vspace{0.5cm}

\begin{tabularx}{\textwidth}{|X|X|}
\hline
\rowcolor{LightGray}
\textbf{Capacités Avancées} & \textbf{Architecture Plugin} \\
\hline
Plateforme moderne de visualisation avec support natif des images pyramidales et outils d'interaction avancés &
\begin{itemize}[nosep]
\item Visualisation images pyramidales
\item Architecture modulaire extensible
\item Outils d'interaction avancés
\item Rendu OpenGL optimisé
\end{itemize} \\
\hline
\end{tabularx}

\vspace{0.5cm}

\textbf{Potentiel technique :}
\begin{itemize}[leftmargin=*]
    \pro{Plateforme moderne et performante}
    \pro{Support natif des images pyramidales}
    \pro{Architecture plugin flexible}
    \pro{Outils d'interaction avancés}
    \pro{Communauté active de développeurs}
\end{itemize}

\textbf{Limitation architecturale bloquante :}
\begin{itemize}[leftmargin=*]
    \con{Un seul fragment visualisable par canvas}
    \con{Impossible de manipuler plusieurs fragments simultanément}
    \con{Architecture non adaptée à la suture multi-fragments}
    \con{Nécessiterait une refonte architecturale majeure}
\end{itemize}

\begin{center}
\textbf{\textcolor{DangerRed}{DÉCISION : REJETÉ}}\\
\textit{Limitations architecturales fondamentales}
\end{center}

\end{techbox}

\subsection{Synthèse Comparative}

\begin{center}
\textbf{Tableau Comparatif des Technologies Évaluées}
\end{center}

\begin{table}[h!]
\centering
\begin{tabularx}{\textwidth}{|l|C{2cm}|C{2cm}|C{2cm}|C{2cm}|C{2cm}|}
\hline
\rowcolor{LightGray}
\textbf{Technologie} & \textbf{Scalabilité} & \textbf{Formats Médicaux} & \textbf{Maintenance} & \textbf{Déploiement} & \textbf{Verdict} \\
\hline
PyThostitcher & \textcolor{DangerRed}{\faTimes} & \textcolor{WarningOrange}{\faExclamationTriangle} & \textcolor{SuccessGreen}{\faCheck} & \textcolor{SuccessGreen}{\faCheck} & \textcolor{DangerRed}{REJETÉ} \\
\hline
HistoStitcher & \textcolor{WarningOrange}{\faExclamationTriangle} & \textcolor{DangerRed}{\faTimes} & \textcolor{DangerRed}{\faTimes} & \textcolor{DangerRed}{\faTimes} & \textcolor{DangerRed}{REJETÉ} \\
\hline
ASHLAR & \textcolor{SuccessGreen}{\faCheck} & \textcolor{SuccessGreen}{\faCheck} & \textcolor{SuccessGreen}{\faCheck} & \textcolor{SuccessGreen}{\faCheck} & \textcolor{WarningOrange}{INADAPTÉ} \\
\hline
FIJI/ImageJ & \textcolor{SuccessGreen}{\faCheck} & \textcolor{WarningOrange}{\faExclamationTriangle} & \textcolor{SuccessGreen}{\faCheck} & \textcolor{SuccessGreen}{\faCheck} & \textcolor{DangerRed}{REJETÉ} \\
\hline
Plugin Napari & \textcolor{DangerRed}{\faTimes} & \textcolor{SuccessGreen}{\faCheck} & \textcolor{SuccessGreen}{\faCheck} & \textcolor{SuccessGreen}{\faCheck} & \textcolor{DangerRed}{REJETÉ} \\
\hline
\end{tabularx}
\end{table}

\subsection{Justification de la Solution Retenue}

\begin{tcolorbox}[colback=SuccessGreen!10, colframe=SuccessGreen, title=Développement Personnalisé - Solution Optimale]

Face aux limitations identifiées dans toutes les solutions existantes, le développement d'un outil personnalisé s'impose comme la seule approche viable.

\vspace{0.5cm}

\textbf{Analyse des Contraintes Spécifiques :}
\begin{itemize}[leftmargin=*]
    \item \textbf{Tissus découpés physiquement} : Absence de zones de chevauchement
    \item \textbf{Nombre variable de fragments} : De 2 à 15+ fragments par cas
    \item \textbf{Formats spécialisés} : SVS, MRXS, TIFF pyramidal obligatoires
    \item \textbf{Environnement clinique} : Sécurité et fiabilité maximales requises
\end{itemize}

\vspace{0.5cm}

\textbf{Avantages Décisifs du Développement Personnalisé :}
\begin{itemize}[leftmargin=*]
    \pro{Performance optimale avec accès direct aux ressources système}
    \pro{Contrôle total sur l'interface utilisateur et l'expérience}
    \pro{Sécurité maximale avec traitement local des données}
    \pro{Adaptation parfaite aux besoins spécifiques du projet TEP Margins}
    \pro{Évolutivité complète selon les retours utilisateurs}
    \pro{Indépendance technologique et maintenance maîtrisée}
\end{itemize}

\vspace{0.5cm}

\textbf{Défis à Relever :}
\begin{itemize}[leftmargin=*]
    \con{Temps de développement élevé (13 semaines)}
    \con{Distribution nécessitant un packaging spécialisé}
    \con{Maintenance et évolution à long terme à prévoir}
    \con{Tests exhaustifs en environnement clinique requis}
\end{itemize}

\end{tcolorbox}

\begin{center}
\textbf{Conclusion de l'Analyse}
\end{center}

L'analyse comparative approfondie des technologies existantes révèle que **aucune solution disponible ne répond aux contraintes spécifiques** du projet TEP Margins. Les outils analysés présentent tous des limitations fondamentales :

\begin{itemize}[leftmargin=*]
    \item \textbf{Contraintes de scalabilité} (PyThostitcher, Napari)
    \item \textbf{Obsolescence technique} (HistoStitcher)
    \item \textbf{Inadéquation algorithmique} (ASHLAR, FIJI)
\end{itemize}

Le **développement d'un outil personnalisé** constitue donc la seule approche permettant de répondre efficacement aux besoins identifiés, tout en garantissant une solution pérenne et évolutive pour le Centre Henri Becquerel.