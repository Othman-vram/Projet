\subsubsection{PyVIPS : Optimisation de l'exportation pyramidale}

PyVIPS est une bibliothèque Python qui fournit des bindings pour VIPS (Vision Image Processing System), un système de traitement d'images conçu spécifiquement pour gérer efficacement les images de très grande taille. Dans le contexte de notre projet, PyVIPS joue un rôle crucial dans le système d'exportation pour la génération d'images TIFF pyramidales de haute qualité à partir des fragments assemblés.

\textbf{Caractéristiques techniques de PyVIPS :}

\begin{itemize}[leftmargin=*]
    \item \textbf{Traitement par flux} : PyVIPS utilise une approche de traitement par flux (streaming) qui ne charge en mémoire que les portions d'image nécessaires au calcul en cours
    \item \textbf{Gestion pyramidale native} : Support intégré des structures pyramidales TIFF avec accès optimisé aux différents niveaux de résolution
    \item \textbf{Parallélisation automatique} : Exploitation automatique des processeurs multi-cœurs pour accélérer les opérations de traitement
    \item \textbf{Optimisation mémoire} : Utilisation d'un cache intelligent et de techniques de pagination pour minimiser l'empreinte mémoire
\end{itemize}

\textbf{Intégration dans notre système d'exportation :}

Dans le module d'exportation, PyVIPS est utilisé pour générer des images TIFF pyramidales composites à partir des fragments assemblés dans l'interface de suture. Le système d'exportation utilise PyVIPS pour créer efficacement les différents niveaux de résolution de l'image finale, en appliquant les transformations géométriques (rotation, translation, retournement) calculées par l'interface utilisateur. Cette approche garantit que la structure pyramidale est correctement générée avec préservation de la qualité d'image à tous les niveaux.

\begin{figure}[H]
\centering
\includegraphics[width=0.9\textwidth]{images/dialogue_export_screenshot.png}
\caption{Interface d'exportation utilisant PyVIPS pour la génération de TIFF pyramidaux avec sélection des niveaux}
\label{fig:pyvips_export}
\end{figure}

\textbf{Avantages décisifs pour notre projet :}

\begin{itemize}[leftmargin=*]
    \item \textbf{Scalabilité} : Capacité à traiter des images de plusieurs gigaoctets sans saturation mémoire
    \item \textbf{Performance} : Temps de traitement optimisés grâce à la parallélisation et au streaming
    \item \textbf{Qualité} : Préservation de la qualité d'image et des métadonnées lors des conversions
    \item \textbf{Compatibilité} : Support natif des formats médicaux (SVS, MRXS) via OpenSlide
\end{itemize}

L'utilisation de PyVIPS dans notre système d'exportation permet de générer efficacement des images TIFF pyramidales composites de haute qualité à partir des fragments assemblés, tout en maintenant des performances acceptables lors de l'exportation d'images de plusieurs gigaoctets sur des postes de travail cliniques standards. Cette approche garantit que les images finales conservent leur structure multi-résolution, essentielle pour l'intégration dans le protocole TEP Margins.