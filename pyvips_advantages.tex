\textbf{Avantages de PyVIPS pour l'exportation pyramidale :}

PyVIPS utilise une approche par tuiles (tile-based processing) qui optimise considérablement l'utilisation de la mémoire lors de l'exportation d'images composites de très haute résolution. Cette technique divise l'image finale en petites tuiles de taille fixe (par exemple 256×256 pixels) et traite chaque tuile individuellement, ne chargeant en mémoire que les portions d'image strictement nécessaires au calcul en cours. Cette approche permet de traiter des images de plusieurs gigaoctets sans saturer la mémoire RAM disponible sur les postes de travail cliniques standards.

Dans le contexte de notre projet, cette optimisation est cruciale car l'exportation d'images composites reconstituées à partir de multiples fragments TIFF pyramidaux peut générer des fichiers finaux de très grande taille. L'approche par tuiles de PyVIPS garantit que le processus d'exportation reste stable et performant même lors de la génération d'images finales de plusieurs gigapixels, permettant ainsi l'utilisation de l'outil sur l'infrastructure informatique existante du Centre Henri Becquerel sans nécessiter de ressources de calcul dédiées.