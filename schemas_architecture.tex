\documentclass[10pt,a4paper]{article}
\usepackage[utf8]{inputenc}
\usepackage[french]{babel}
\usepackage{tikz}
\usepackage{geometry}
\geometry{margin=2cm}

% Définition des couleurs
\definecolor{inputcolor}{RGB}{52, 152, 219}
\definecolor{processcolor}{RGB}{46, 204, 113}
\definecolor{outputcolor}{RGB}{231, 76, 60}
\definecolor{toolcolor}{RGB}{155, 89, 182}
\definecolor{datacolor}{RGB}{241, 196, 15}

% Styles des nœuds
\tikzstyle{input} = [rectangle, rounded corners, minimum width=3cm, minimum height=1cm, text centered, draw=inputcolor, fill=inputcolor!20, thick]
\tikzstyle{process} = [rectangle, rounded corners, minimum width=3cm, minimum height=1cm, text centered, draw=processcolor, fill=processcolor!20, thick]
\tikzstyle{output} = [rectangle, rounded corners, minimum width=3cm, minimum height=1cm, text centered, draw=outputcolor, fill=outputcolor!20, thick]
\tikzstyle{tool} = [rectangle, rounded corners, minimum width=2cm, minimum height=0.7cm, text centered, draw=toolcolor, fill=toolcolor!20, thick]
\tikzstyle{data} = [ellipse, minimum width=2cm, minimum height=0.8cm, text centered, draw=datacolor, fill=datacolor!20, thick]
\tikzstyle{arrow} = [thick,->,>=stealth]
\tikzstyle{dasharrow} = [thick,->,>=stealth,dashed]

\begin{document}

\title{Schémas d'Architecture Technique\\Outil de Suture d'Images Histologiques}
\author{EL IDRISSI Othman}
\date{\today}

\maketitle

\section{Architecture de la Phase de Prétraitement}

\begin{figure}[htbp]
\centering
\begin{tikzpicture}[node distance=1.5cm, scale=0.8, every node/.style={scale=0.8}]

% Titre du schéma
\node[above] at (0,6.5) {\large \textbf{Pipeline de Prétraitement des Images Histologiques}};

% Entrées
\node (svs) [input] at (-5,5) {Image SVS/MRXS\\(Lame scannée)};
\node (qupath) [tool] at (-2,5) {QuPath\\+ Plugin SAM};

% Processus de segmentation
\node (selection) [process] at (1.5,5) {Sélection\\Zones de Tissu};
\node (geojson) [data] at (4.5,5) {Masque\\GeoJSON};

% Pipeline Python
\node (pipeline) [process] at (0,3) {Pipeline Python\\unified\_tissue\_pipeline.py};

% Sous-processus du pipeline
\node (conversion) [process] at (-3,1) {Conversion\\SVS → TIFF};
\node (maskgen) [process] at (0,1) {Génération\\Masque Pyramidal};
\node (extraction) [process] at (3,1) {Extraction\\Tissus RGBA};

% Sortie finale
\node (tiffout) [output] at (0,-0.5) {TIFF Pyramidal\\RGBA Prétraité};

% Flèches principales
\draw [arrow] (svs) -- (qupath);
\draw [arrow] (qupath) -- (selection);
\draw [arrow] (selection) -- (geojson);

% Flèches vers le pipeline
\draw [arrow] (svs) -- (-2.5,3.8) -- (-1,3);
\draw [arrow] (geojson) -- (2.5,3.8) -- (1,3);

% Flèches du pipeline
\draw [arrow] (pipeline) -- (conversion);
\draw [arrow] (pipeline) -- (maskgen);
\draw [arrow] (pipeline) -- (extraction);

% Flèches entre sous-processus
\draw [arrow] (conversion) -- (maskgen);
\draw [arrow] (maskgen) -- (extraction);

% Flèche finale
\draw [arrow] (extraction) -- (tiffout);

% Annotations
\node[text width=2.5cm, align=center] at (-6.5,2.5) {\tiny \textit{Formats supportés:\\SVS, MRXS, TIFF}};
\node[text width=2.5cm, align=center] at (6.5,2.5) {\tiny \textit{Segmentation\\automatique\\avec SAM}};
\node[text width=3cm, align=center] at (4.5,-0.5) {\tiny \textit{Fond transparent\\Tissus préservés\\Multi-résolution}};

% Cadre pour le pipeline
\draw[dashed, thick, processcolor] (-4,0.3) rectangle (4,1.7);
\node[processcolor] at (0,1.9) {\small \textbf{Étapes du Pipeline Unifié}};

\end{tikzpicture}
\caption{Architecture de la phase de prétraitement}
\end{figure}

\newpage

\section{Architecture de la Phase de Suture}

\begin{figure}[htbp]
\centering
\begin{tikzpicture}[node distance=1.5cm, scale=0.75, every node/.style={scale=0.75}]

% Titre du schéma
\node[above] at (0,7.5) {\large \textbf{Architecture de l'Application de Suture Manuelle}};

% Entrée
\node (tiffinput) [input] at (0,6.5) {Fragments TIFF\\Pyramidaux Prétraités};

% Couche de chargement
\node (loader) [process] at (0,5.5) {Image Loader\\(OpenSlide + tifffile)};

% Architecture MVC
\node (mvc) [above] at (0,4.3) {\small \textbf{Architecture MVC (PyQt6)}};

% Modèle
\node (fragmanager) [process] at (-3.5,3.5) {Fragment\\Manager};
\node (pointmanager) [process] at (-3.5,2.5) {Point\\Manager};

% Vue
\node (mainwindow) [process] at (0,3.5) {Main Window};
\node (canvas) [process] at (0,2.5) {Canvas Widget\\(Visualisation)};
\node (controls) [process] at (0,1.5) {Control Panel\\(Manipulation)};

% Contrôleur
\node (algorithms) [process] at (3.5,3.5) {Algorithmes\\de Suture};
\node (exporters) [process] at (3.5,2.5) {Export\\Manager};

% Fonctionnalités principales
\node (features) [above] at (0,0.3) {\small \textbf{Fonctionnalités Principales}};

\node (manipulation) [tool] at (-2.5,-0.5) {Manipulation\\Fragments};
\node (alignment) [tool] at (0,-0.5) {Alignement\\Manuel};
\node (stitching) [tool] at (2.5,-0.5) {Suture\\Rigide};

% Sortie
\node (finaloutput) [output] at (0,-2) {Image Finale\\TIFF Pyramidal\\Haute Définition};

% Flèches principales
\draw [arrow] (tiffinput) -- (loader);
\draw [arrow] (loader) -- (mainwindow);

% Flèches MVC
\draw [arrow] (fragmanager) -- (mainwindow);
\draw [arrow] (pointmanager) -- (mainwindow);
\draw [arrow] (mainwindow) -- (canvas);
\draw [arrow] (mainwindow) -- (controls);
\draw [arrow] (algorithms) -- (mainwindow);
\draw [arrow] (exporters) -- (mainwindow);

% Flèches vers fonctionnalités
\draw [arrow] (controls) -- (manipulation);
\draw [arrow] (controls) -- (alignment);
\draw [arrow] (algorithms) -- (stitching);

% Flèche finale
\draw [arrow] (exporters) -- (finaloutput);

% Annotations techniques
\node[text width=2cm, align=center] at (-5.5,2.5) {\tiny \textit{Gestion état\\fragments\\transformations}};
\node[text width=2cm, align=center] at (5.5,2.5) {\tiny \textit{SIFT features\\Optimisation\\L-BFGS-B}};
\node[text width=2.5cm, align=center] at (4.5,-2) {\tiny \textit{Multi-niveaux\\Compression LZW\\Métadonnées}};

% Cadres pour grouper les composants
\draw[dashed, thick, processcolor] (-4.5,2) rectangle (-2.5,4);
\node[processcolor] at (-3.5,4.2) {\small \textbf{Modèle}};

\draw[dashed, thick, toolcolor] (-1,1) rectangle (1,4);
\node[toolcolor] at (0,4.2) {\small \textbf{Vue}};

\draw[dashed, thick, outputcolor] (2.5,2) rectangle (4.5,4);
\node[outputcolor] at (3.5,4.2) {\small \textbf{Contrôleur}};

\end{tikzpicture}
\caption{Architecture de l'application de suture manuelle}
\end{figure}

\newpage

\section{Flux de Données Global}

\begin{figure}[htbp]
\centering
\begin{tikzpicture}[node distance=1.3cm, scale=0.8, every node/.style={scale=0.8}]

% Titre
\node[above] at (0,6.5) {\large \textbf{Flux de Données Global du Système}};

% Phase 1 - Acquisition
\node (microscope) [input] at (-4.5,5) {Microscope\\Scanner};
\node (rawimage) [data] at (-2,5) {Image Brute\\SVS/MRXS};

% Phase 2 - Prétraitement
\node (preprocessing) [process] at (0.5,5) {Prétraitement\\Pipeline};
\node (fragments) [data] at (3,5) {Fragments\\TIFF RGBA};

% Phase 3 - Suture
\node (stitching) [process] at (0.5,3) {Application\\Suture};
\node (finalimage) [output] at (3,3) {Image Finale\\Reconstituée};

% Phase 4 - Utilisation
\node (analysis) [tool] at (0.5,1) {Analyse\\TEP Margins};
\node (diagnosis) [output] at (3,1) {Diagnostic\\Médical};

% Flèches principales
\draw [arrow] (microscope) -- (rawimage);
\draw [arrow] (rawimage) -- (preprocessing);
\draw [arrow] (preprocessing) -- (fragments);
\draw [arrow] (fragments) -- (stitching);
\draw [arrow] (stitching) -- (finalimage);
\draw [arrow] (finalimage) -- (analysis);
\draw [arrow] (analysis) -- (diagnosis);

% Annotations de phases
\node[text width=1.5cm, align=center] at (-4.5,3.5) {\small \textbf{Phase 1}\\Acquisition};
\node[text width=1.5cm, align=center] at (0.5,3.8) {\small \textbf{Phase 2}\\Prétraitement};
\node[text width=1.5cm, align=center] at (0.5,2) {\small \textbf{Phase 3}\\Suture};
\node[text width=1.5cm, align=center] at (0.5,0.2) {\small \textbf{Phase 4}\\Application};

% Détails techniques
\node[text width=2.5cm, align=center] at (-4.5,6) {\tiny \textit{Résolution:\\0.25 µm/pixel\\Format propriétaire}};
\node[text width=2.5cm, align=center] at (5,5) {\tiny \textit{Fond transparent\\Multi-résolution\\Optimisé}};
\node[text width=2.5cm, align=center] at (5,3) {\tiny \textit{Haute définition\\Métadonnées\\Exportable}};
\node[text width=2.5cm, align=center] at (5,1) {\tiny \textit{Marges chirurgicales\\Micro-TEP\\Corrélation}};

\end{tikzpicture}
\caption{Flux de données global du système}
\end{figure}

\section{Diagramme de Classes Simplifié}

\begin{figure}[htbp]
\centering
\begin{tikzpicture}[node distance=1.8cm, scale=0.7, every node/.style={scale=0.7}]

% Titre
\node[above] at (0,6) {\large \textbf{Diagramme de Classes Principal}};

% Classes principales
\node (fragment) [process, text width=2.8cm] at (-4,4) {
    \textbf{Fragment}\\
    \rule{2.8cm}{0.4pt}\\
    - id: str\\
    - image\_data: ndarray\\
    - x, y: float\\
    - rotation: float\\
    - visible: bool\\
    \rule{2.8cm}{0.4pt}\\
    + get\_transformed\_image()\\
    + contains\_point()
};

\node (fragmanager) [process, text width=2.8cm] at (0,4) {
    \textbf{FragmentManager}\\
    \rule{2.8cm}{0.4pt}\\
    - fragments: Dict\\
    - selected\_id: str\\
    \rule{2.8cm}{0.4pt}\\
    + add\_fragment()\\
    + rotate\_fragment()\\
    + translate\_fragment()
};

\node (canvas) [process, text width=2.8cm] at (4,4) {
    \textbf{CanvasWidget}\\
    \rule{2.8cm}{0.4pt}\\
    - zoom: float\\
    - pan\_x, pan\_y: float\\
    - fragments: List\\
    \rule{2.8cm}{0.4pt}\\
    + paintEvent()\\
    + mousePressEvent()
};

\node (imageloader) [process, text width=2.8cm] at (-4,1.5) {
    \textbf{ImageLoader}\\
    \rule{2.8cm}{0.4pt}\\
    - supported\_formats\\
    \rule{2.8cm}{0.4pt}\\
    + load\_image()\\
    + get\_pyramid\_info()
};

\node (exporter) [process, text width=2.8cm] at (4,1.5) {
    \textbf{PyramidalExporter}\\
    \rule{2.8cm}{0.4pt}\\
    \rule{2.8cm}{0.4pt}\\
    + export\_pyramidal\_tiff()\\
    + create\_level\_composite()
};

% Relations
\draw [arrow] (fragmanager) -- (fragment);
\draw [arrow] (canvas) -- (fragmanager);
\draw [arrow] (imageloader) -- (fragment);
\draw [arrow] (exporter) -- (fragmanager);

% Labels des relations
\node at (-2,4.2) {\tiny manages};
\node at (2,4.2) {\tiny displays};
\node at (-4,2.8) {\tiny creates};
\node at (4,2.8) {\tiny exports};

\end{tikzpicture}
\caption{Diagramme de classes simplifié}
\end{figure}

\end{document}