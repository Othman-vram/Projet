\documentclass[11pt,a4paper]{report}
\usepackage[utf8]{inputenc}
\usepackage[T1]{fontenc}
\usepackage[french]{babel}
\usepackage{geometry}
\usepackage{graphicx}
\usepackage{amsmath}
\usepackage{amsfonts}
\usepackage{amssymb}
\usepackage{fancyhdr}
\usepackage{titlesec}
\usepackage{tocloft}
\usepackage{hyperref}
\usepackage{listings}
\usepackage{xcolor}
\usepackage{float}
\usepackage{caption}
\usepackage{subcaption}
\usepackage{enumitem}
\usepackage{array}
\usepackage{longtable}
\usepackage{booktabs}
\usepackage{multirow}
\usepackage{url}
\usepackage{cite}

% Configuration de la page
\geometry{left=2.5cm,right=2.5cm,top=2.5cm,bottom=2.5cm}

% Configuration des en-têtes et pieds de page
\pagestyle{fancy}
\fancyhf{}
\fancyhead[L]{\leftmark}
\fancyhead[R]{\thepage}
\renewcommand{\headrulewidth}{0.4pt}

% Configuration des titres
\titleformat{\chapter}[display]
{\normalfont\huge\bfseries}{\chaptertitlename\ \thechapter}{20pt}{\Huge}
\titlespacing*{\chapter}{0pt}{-30pt}{40pt}

% Configuration des liens
\hypersetup{
    colorlinks=true,
    linkcolor=black,
    filecolor=magenta,      
    urlcolor=blue,
    citecolor=red,
}

% Configuration du code
\lstset{
    basicstyle=\ttfamily\footnotesize,
    breaklines=true,
    frame=single,
    numbers=left,
    numberstyle=\tiny,
    showstringspaces=false,
    commentstyle=\color{gray},
    keywordstyle=\color{blue},
    stringstyle=\color{red}
}

\begin{document}

% Page de garde
\begin{titlepage}
\centering
\vspace*{1cm}

{\huge\bfseries INSA Rouen Normandie}\\[0.5cm]
{\large Département Sciences et Technologies de l'Information}\\[1.5cm]

{\Large\bfseries RAPPORT DE STAGE DE SPÉCIALITÉ}\\[1cm]

{\huge\bfseries Création et Mise en Œuvre d'un Outil de Génération d'Images d'Anatomopathologie de Haute Définition à partir de Lames Scannées}\\[2cm]

{\large Stage effectué du 2 juin 2025 au 31 août 2025}\\[0.5cm]
{\large au Centre Henri Becquerel, Rouen}\\[2cm]

\begin{minipage}{0.4\textwidth}
\begin{flushleft}
{\large\bfseries Stagiaire :}\\
EL IDRISSI Othman\\
Élève ingénieur 4ème année\\
Spécialité Informatique et Technologies de l'Information
\end{flushleft}
\end{minipage}
\hfill
\begin{minipage}{0.4\textwidth}
\begin{flushright}
{\large\bfseries Encadrement :}\\
Tuteur entreprise :\\
M. Sébastien HAPDEY\\
Physicien Médical\\[0.5cm]
Enseignant référent :\\
M. Benoît GAUZÈRE
\end{flushright}
\end{minipage}

\vfill

{\large Centre Henri Becquerel}\\
{\large Rue d'Amiens, CS 11516}\\
{\large 76038 Rouen Cedex 1, France}\\
{\large Haute-Normandie}

\end{titlepage}

% Remerciements
\chapter*{Remerciements}
\addcontentsline{toc}{chapter}{Remerciements}

Je tiens à adresser ma profonde gratitude à M. Sébastien HAPDEY, physicien médical et tuteur pédagogique au Centre Henri Becquerel de Rouen. Son suivi attentif, ses conseils pertinents et son encadrement bienveillant ont été déterminants pour le bon déroulement de ce stage. Sa disponibilité et sa capacité à orienter mon travail avec justesse m'ont permis de progresser et d'acquérir des connaissances solides dans ce domaine exigeant.

Je souhaite également exprimer mes sincères remerciements à M. Romain MODZELEWSKI, responsable informatique biomédicale au département d'imagerie – Laboratoire AIMS-Quantif. Ses explications claires, son expertise technique et son sens du partage ont constitué un appui essentiel pour la réussite de ce projet. Son engagement et sa réactivité ont grandement facilité la réalisation des différentes étapes de mon travail.

Mes remerciements s'adressent enfin à l'ensemble du Centre Henri Becquerel, dont l'accueil chaleureux, l'organisation et les conditions de travail favorables ont contribué à rendre cette expérience formatrice et enrichissante.

% Table des matières
\tableofcontents
\newpage

% Liste des figures
\listoffigures
\newpage

% Liste des tableaux
\listoftables
\newpage

% Introduction
\chapter*{Introduction}
\addcontentsline{toc}{chapter}{Introduction}

L'anatomopathologie moderne fait face à un défi technique majeur : comment traiter efficacement les images histologiques fragmentées ? Cette problématique, que j'ai découverte lors de mon stage au Centre Henri Becquerel, m'a amené à développer une solution informatique innovante.

Quand les lames histologiques sont numérisées, il arrive fréquemment que les tissus se fragmentent ou se déplacent pendant le processus de scan. Résultat ? Des images incomplètes qui compliquent le diagnostic médical. C'est exactement le problème que nous avons cherché à résoudre.

Mon stage de 13 semaines s'est donc concentré sur la création d'un outil permettant de "recoller" numériquement ces fragments. Pas si simple quand on découvre la complexité des formats d'images médicales et les exigences de précision requises !

Ce rapport est organisé en quatre chapitres :
\begin{itemize}
\item D'abord, je présenterai le Centre Henri Becquerel et le contexte médical du projet
\item Ensuite, j'expliquerai en détail le problème technique à résoudre
\item Puis, je détaillerai tout le travail de développement réalisé
\item Enfin, je ferai un bilan critique de cette expérience
\end{itemize}

Ce projet m'a vraiment fait découvrir les enjeux de l'informatique médicale, un domaine où la moindre erreur peut avoir des conséquences importantes.

\chapter{Présentation de l'entreprise et de l'environnement du stage}

\section{Le Centre Henri-Becquerel}

Le Centre Henri-Becquerel est un Centre de Lutte Contre le Cancer (CLCC) situé à Rouen, en France. Faisant partie du réseau national Unicancer, il assure une triple mission de soins, de recherche et d'enseignement. Il prend en charge la majorité des pathologies cancéreuses et dispose d'un plateau technique intégré comprenant la radiothérapie, la médecine nucléaire et la radiologie. Le Centre est également labellisé « OECI » par l'Association Européenne des Centres Anti-Cancer.

Le Centre Henri-Becquerel se distingue par son approche multidisciplinaire de la prise en charge du cancer, intégrant les dernières avancées technologiques et scientifiques. Cette philosophie se traduit par une recherche constante d'innovation dans les domaines de l'imagerie médicale, de la radiothérapie et de l'anatomopathologie numérique.

\section{L'équipe QuantIF}

C'est au sein de l'équipe QuantIF que j'ai effectué mon stage. Cette équipe de recherche, rattachée au laboratoire LITIS, travaille sur des sujets passionnants : comment améliorer le diagnostic médical grâce aux nouvelles technologies d'imagerie ? Leur spécialité ? Les cancers du thorax et de l'abdomen, des pathologies où chaque détail compte.

Cette équipe constitue un environnement de recherche particulièrement stimulant, où se côtoient médecins, physiciens, informaticiens et ingénieurs. Cette diversité disciplinaire favorise l'émergence de solutions innovantes et l'application concrète des avancées technologiques aux problématiques cliniques.

\section{Thèmes et axes de recherche}

Les recherches de l'équipe QuantIF se basent sur plusieurs modalités d'imagerie :

\begin{itemize}
\item Le couplage Tomographie par Émission de Positons / TomoDensitoMétrie (TEP/TDM)
\item L'imagerie microendoscopique confocale fibrée (imagerie en fluorescence)
\item L'Imagerie par Résonance Magnétique (IRM)
\end{itemize}

De ces modalités découlent trois questions médicales d'intérêt :

\begin{itemize}
\item L'amélioration du ciblage et de la balistique du cancer pulmonaire en radiothérapie grâce à l'imagerie fonctionnelle TEP/TDM (responsabilité : Pr Vera)
\item La caractérisation de l'alvéole pulmonaire grâce aux nouvelles techniques d'imagerie microendoscopique confocale (responsabilité : Pr Thiberville)
\item La caractérisation du foie et du tube digestif en IRM (responsabilité : Pr Savoye-Collet)
\end{itemize}

Les verrous en traitement d'images sont la classification et la sélection de caractéristiques. Les travaux portent également sur l'amélioration des données quantitatives des images, leur segmentation et la fusion d'informations.

\section{Composition de l'équipe et plateau technique}

L'équipe est composée de 15 membres permanents et de 7 doctorants :

\begin{itemize}
\item \textbf{4 PU-PH} : B. DUBRAY, L. THIBERVILLE, P. VERA, C. SAVOYE-COLLET
\item \textbf{1 PU} : S. RUAN
\item \textbf{2 MCU-PH} : JF. MENARD, M. SALAÜN
\item \textbf{2 MdC} : C. PETITJEAN, J. LAPUYADE
\item \textbf{6 PH} : S. BECKER, A. EDET-SANSON, I. GARDIN, S. HAPDEY, P. BOHN, R. MODZELEWSKI
\item \textbf{1 Ingénieur} : R. MODZELEWSKI
\end{itemize}

Pour ses travaux, l'équipe dispose des équipements suivants : une plateforme d'imagerie du petit animal, un laboratoire de traitement d'image, ainsi que l'accès aux équipements d'imagerie des CHU et CHB (IRM, TDM, TEP-TDM, etc.) et au service de radiothérapie du CHB.

\chapter{Présentation du sujet du stage}

\section{Contexte général}

Imaginez un chirurgien qui opère un cancer de la gorge. Sa préoccupation principale ? S'assurer qu'il a re